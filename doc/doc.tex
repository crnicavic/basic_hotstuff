\documentclass[12pt]{article}                                                   
\usepackage{tikz}
\usetikzlibrary{shapes,arrows}
\usepackage{anyfontsize}
\usepackage{microtype}
\usepackage{hyperref}
\usepackage{indentfirst}
\usepackage{placeins}
\usepackage{mathtools}                                                          
\usepackage{amsmath}
\usepackage[labelformat=simple]{subfig}
\usepackage{tabularx}
\usepackage{fontspec}
\usepackage{caption}
\usepackage{geometry} 
\usepackage{float}
	\geometry{
		a4paper,
		lmargin=3cm,
		rmargin=2cm,
		tmargin=2cm,
		bmargin=2cm
	}
\renewcommand{\contentsname}{Sadržaj}
\setmainfont[Mapping=tex-text]{Times New Roman}
\renewcommand{\thesubfigure}{\relax}  % Do nothing for the counter »subfigure«
\newlength{\plotheight}
\setlength{\plotheight}{4.5cm}
\def\plotscale{0.65}
\begin{document}

\begin{titlepage}
	\begin{center}
        \vspace*{1cm}
        \Large
        \textbf{Univerzitet u Novom Sadu\\}
        \vspace{0.3cm}
        \Large
        Fakultet tehničkih nauka
        \vfill
		\Huge
		\textbf{Implementacija i analiza \\ ponašanja Hotstuff BFT \\ konsenzus algoritma}
        \vfill
        \large
        Ognjen Čavić\\
        Novi Sad, februar 2026.
    \end{center}
\end{titlepage}
\tableofcontents
\break
\section{Uvod}
Savremeni informacioni sistemi u velikoj meri se oslanjaju na distribuirane
sisteme, u kojima više računarskih čvorova sarađuje kako bi obezbedili pouzdano
i efikasno izvršavanje zadataka. Ovakvi sistemi omogućuju skalabilnost,
paralelno izvršavanje i visoku dostupnost, što ih čini osnovom velikog broja
današnjih softverskih rešenja.

Jedan od osnovnih pristupa za povećanje pouzdanosti distribuiranih sistema
jeste replikacija stanja između više čvorova. Ovaj princip formalizovan je kroz
koncept \textbf{replikacije mašine stanja} (\textit{State machine replication - SMR}),
prema kome sve replike u sistemu izvršavaju iste operacije u istom redosledu,
čime se obezbeđuje konzistentno ponašanje sistema i tolerancija na pojedinačne
otkaze.

Da bi SMR bio realizovan u realnim uslovima, neophodno je obezbediti saglasnost
između replika o redosledu izvršavanja operacija. Ovaj proces mora garantovati
bezbednost (\textit{safety}), u smislu očuvanja konzistentnosti sistema, kao i živost
(\textit{liveness}), odnosno mogućnost da sistem nastavi sa radom u prisustvu grešaka.
Ovi zahtevi postaju naročito složeni u prisustvu vizantijskih otkaza, kada se
pojedini čvorovi mogu ponašati proizvoljno, uključujući prestanak rada ili
slanje kontradiktornih informacija drugim učesnicima sistema.

Konsenzus algoritmi opisuju način na koji se čvorovi dogovaraju o redosledu
izvršavanja operacija. Ukoliko algoritam omogućava sistemu da nastavi sa radom 
i u prisustvu vizantijskih otkaza, za algoritam se može reći da je tolerantan
na vizantijske otkaze (Byzantine fault tolerance - BFT).

\textbf{Hotstuff} je BFT konsenzus algoritam zasnovan na vođi (\textit{leader-based}) 
koji postiže saglasnost između replika u parcijalno sinhronom mrežnom modelu.
Kada mrežna komunikacija postane sinhrona, vođa vodi dostizanje dogovora brzinom
koja je uslovljena karakteristikama mrežne komunikacije, što je osobina koja se
naziva odzivnost (\textit{responsiveness}). Komunikaciona složenost algoritma je $O(n)$,
gde je $n$ broj replika u sistemu, dok u slučaju uzastopnih otkaza vođe,
složenost u najgorem slučaju može dostići $O(n^2)$.

Ovaj rad obuhvata analizu ključnih principa rada HotStuff algoritma i njegovih
fundamentalnih struktura podataka, koje omogućuju pouzdano postizanje
konsenzusa u okviru SMR modela sa vizantijskim otkazima. Na osnovu izloženog
teorijskog okvira razvijena je implementacija algoritma i sprovedena evaluacija
njegovih karakteristika kroz skup eksperimenata.
Rezultati ovih eksperimenata korišćeni su za procenu ispravnosti rada algoritma,
njegove efikasnosti i ponašanja u slučaju vizantijskih otkaza.
\newpage
\section{Svojstva algoritma}
\subsection{Bezbednost i živost}
Osnovni zahtevi koje konsenzus algoritam u okviru SMR mora da ispuni jesu
bezbednost i živost. Bezbednost podrazumeva da sve korektne replike izvršavaju
operacije u istom redosledu, odnosno da ne može doći do divergentnog stanja
sistema. Drugim rečima, jednom doneta odluka ne sme biti promenjena niti
protivrečiti nekoj drugoj odluci donesene od strane korektnih replika.

Živost označava sposobnost sistema da nastavi sa radom i da donosi nove odluke
uprkos prisustvu grešaka i kašnjenja u komunikaciji. Formalno, živost garantuje
da će svaka ispravna operacija predložena od strane klijenta u konačnom vremenu
biti izvršena, pod pretpostavkom da mrežni uslovi postanu dovoljno povoljni.

Ova dva svojstva predstavljaju temelj SMR modela i ne mogu se razmatrati
nezavisno: protokol koji obezbeđuje samo bezbednost bez živosti dovodi do 
zastoja sistema, dok protokol koji garantuje živost bez bezbednosti ugrožava
konzistentnost stanja replika.
\subsection{Dostizanje konsenzusa zasnovano na vođi}
U velikom broju savremenih konsenzus algoritama proces postizanja saglasnosti
organizovan je kroz ulogu vođe, koja se periodično dodeljuje jednom od
čvorova u sistemu u okviru diskretnih logičkih intervala poznatih kao pogledi
ili mandati (\textit{views}). Tokom jednog pogleda, izabrani vođa je odgovoran za
iniciranje predloga narednih operacija, kao i za koordinaciju glasanja, tj.
prikupljanje odgovora na predloženu operaciju.

Ovakva organizacija izvršavanja  omogućava smanjenje komunikacione složenosti,
jer se predlozi distribuiraju iz jednog centralnog izvora ka ostalim
replikama, koje svoje glasove šalju vođi. Na taj način se izbegava situacija u
kojoj više replika istovremeno predlaže konkurentne operacije, što bi dovelo do
povećanog broja razmenjenih poruka i složenije sinhronizacije sistema.

Uloga vođe nije trajna i vezana je isključivo za jedan pogled. U slučaju da je
vođa spor, postane nedostupan ili se ponaša nekorektno, sistem
prelazi u naredni pogled u kome druga replika preuzima ulogu vođe. Mehanizam
promene pogleda predstavlja osnovni način očuvanja živosti protokola i sprečava
da neispravno ponašanje pojedinačnog čvora trajno blokira napredak sistema.

Mana ovog pristupa je u tome što vođa predstavlja jedinstvenu tačku zastoja,
budući da bilo kakvi otkazi u funkcionisanju vođe izazivaju prestanak rada
algoritma i zahteva prelazak u sledeći pogled gde će biti odabran drugi vođa.
Međutim, jedna od glavnih prednosti Hotstuff algoritma jeste brz i jednostavan
mehanizam promene vođe, čime se omogućuje očuvanje bezbednosti i živosti.

\subsection{Linearna promena pogleda (\textit{Linear view change})}
\subsection{Optimistična odzivnost (\textit{Optimistic responsiveness})}
\end{document}
