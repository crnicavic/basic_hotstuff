\documentclass[12pt]{article}                                                   
\usepackage{tikz}
\usetikzlibrary{shapes,arrows}
\usepackage{anyfontsize}
\usepackage{microtype}
\usepackage{hyperref}
\usepackage{indentfirst}
\usepackage{placeins}
\usepackage{mathtools}                                                          
\usepackage{amsmath}
\usepackage[labelformat=simple]{subfig}
\usepackage{tabularx}
\usepackage{fontspec}
\usepackage{caption}
\usepackage{geometry} 
\usepackage{float}
	\geometry{
		a4paper,
		lmargin=3cm,
		rmargin=2cm,
		tmargin=2cm,
		bmargin=2cm
	}
\renewcommand{\contentsname}{Sadržaj}
\setmainfont[Mapping=tex-text]{Times New Roman}
\renewcommand{\thesubfigure}{\relax}  % Do nothing for the counter »subfigure«
\newlength{\plotheight}
\setlength{\plotheight}{4.5cm}
\def\plotscale{0.65}
\begin{document}

\begin{titlepage}
	\begin{center}
        \vspace*{1cm}
        \Large
        \textbf{Univerzitet u Novom Sadu\\}
        \vspace{0.3cm}
        \Large
        Fakultet tehničkih nauka
        \vfill
		\Huge
		\textbf{Implementacija i analiza \\ ponašanja Hotstuff BFT \\ konsenzus algoritma}
        \vfill
        \large
        Ognjen Čavić\\
        Novi Sad, februar 2026.
    \end{center}
\end{titlepage}
\tableofcontents
\break
\section{Uvod}
Savremeni informacioni sistemi u velikoj meri se oslanjaju na distribuirane
sisteme, u kojima više računarskih čvorova sarađuje kako bi obezbedili pouzdano
i efikasno izvršavanje zadataka. Ovakvi sistemi omogućuju skalabilnost,
paralelno izvršavanje i visoku dostupnost, što ih čini osnovom velikog broja
današnjih softverskih rešenja.

Jedan od osnovnih pristupa za povećanje pouzdanosti distribuiranih sistema
jeste replikacija stanja između više čvorova. Ovaj princip formalizovan je kroz
koncept \textbf{replikacije mašine stanja} (State machine replication - SMR),
prema kome sve replike u sistemu izvršavaju iste operacije u istom redosledu,
čime se obezbeđuje konzistentno ponašanje sistema i tolerancija na pojedinačne
otkaze.

Da bi SMR bio realizovan u realnim uslovima, neophodno je obezbediti
saglasnost između replika o redosledu izvršavanja operacija. Ovaj zadatak
postaje posebno složen u prisustvu vizantijskih otkaza, kada se pojedini
čvorovi mogu ponašati proizvoljno, uključujući prestanak sa radom ili slanjem
kontradiktornih informacija drugim učesnicima.

Konsenzus algoritmi opisuju način na koji se čvorovi dogovaraju o redosledu
izvršavanja operacija. Ukoliko algoritam omogućava sistemu da nastavi sa radom 
i u prisustvu vizantijskih otkaza, za algoritam se može reći da je tolerantan
na vizantijske otkaze (Byzantine fault tolerance - BFT).

\textbf{Hotstuff} je BFT konsenzus algoritam zasnovan na vođi (leader-based) 
koji postiže saglasnost između replika u parcijalno sinhronom mrežnom modelu.
Kada mrežna komunikacija postane sinhrona, vođa vodi dostizanje dogovora brzinom
koja je uslovljena karakteristikama mrežne komunikacije, što je osobina koja se
naziva odzivnost (responsiveness). Komunikaciona složenost algoritma je $O(n)$,
gde je $n$ broj replika u sistemu, dok u slučaju uzastopnih otkaza vođe,
složenost u najgorem slučaju može dostići $O(n^2)$.

Osnovni princip rada algoritma se zasniva na tome da vođa šalje predlog naredne
operacije svim replikama, potom one mogu glasati za taj predlog
ili ostati uzdržane. Ukoliko se prikupi dovoljan broj glasova, vođa formira dokaz
o dovoljnom broju glasova, poznat kao sertifikat kvoruma (Quorum certificate - QC).
Nakon toga slede još dve runde glasanja, prva služi za distribuciju sertifikata
kvoruma, dok druga služi da se naglasi da je operacija spremna za izvršavanje.
Konačno se ta operacija izvršava, što se evidentira u skup izvršenih operacija.

Ovaj rad obuhvata analizu ključnih principa rada HotStuff algoritma i njegovih
fundamentalnih struktura podataka, koje omogućuju pouzdano postizanje
konsenzusa u okviru SMR modela sa vizantijskim otkazima. Na osnovu izloženog
teorijskog okvira razvijena je implementacija algoritma i sprovedena evaluacija
njegovih karakteristika kroz skup eksperimenata.
Rezultati ovih eksperimenata korišćeni su za procenu ispravnosti rada algoritma,
njegove efikasnosti i ponašanja u slučaju vizantijskih otkaza.
\newpage
\section{Svojstva protokola i opšti pojmovi}
\subsection{Bezbednost i živost}

\end{document}
